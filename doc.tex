\documentclass{article}
\usepackage[utf8]{inputenc}
\usepackage{hyperref}
\usepackage{amsmath}
\usepackage{amsfonts}

\begin{document}

\noindent\rule{\textwidth}{1pt} % Linha superior

\begin{center}
    \Large \textbf{Barium: Reinventing Human-Machine Interaction through Advanced Gesture Recognition} % Título
\end{center}

\noindent\rule{\textwidth}{3pt} % Linha inferior

\bigskip % Espaçamento extra após a linha

\begin{center}
   \begin{tabular}{p{0.3\textwidth}p{0.3\textwidth}p{0.3\textwidth}}
        \centering \textbf{Daniel R. Alvarenga*} & \centering \textbf{Alvaro Richard*} & \centering \textbf{Vitor Eduardo S. de Carvalho*} \tabularnewline
        \centering \href{mailto:danielralvs@proton.me}{danielralvs@proton.me} & \centering \href{mailto:alvarorichards@proton.me}{alvaro10048@proton.me} & \centering \href{mailto:srcarvalhox@proton.me}{srcarvalhox@proton.me} \tabularnewline
    \end{tabular}
\end{center}

\bigskip % Espaçamento extra após os autores

\begin{abstract}
The "Barium" project emerges as a pioneering endeavor in the realm of Human-Machine Interfaces (HMI), harnessing the prowess of neural networks, machine learning, and deep learning to track and interpret human body movements, notably hand gestures. This paper delineates the development and application of a 4D neural network training approach, where time is regarded as a crucial dimension, heralding groundbreaking prospects in diverse technological domains. Developed in Python, Barium activates through webcam-captured hand gestures, facilitating user interactions with operating systems via predefined actions and a virtual mouse.
\end{abstract}

% Início da seção dos modelos de rede neural

% \section{Modelos de Rede Neural}
% Para o desenvolvimento deste projeto, foram elaborados dois modelos avançados de rede neural. O primeiro modelo emprega camadas de convolução tridimensional (Conv3D), que são excepcionalmente adequadas para o processamento de dados de vídeo, devido à sua capacidade de capturar características tanto espaciais quanto temporais. O segundo modelo, de natureza sequencial, é focado em otimizar a performance. Este modelo incorpora o cerne científico do projeto, demonstrando uma abordagem inovadora e eficiente no processamento e análise de dados em vídeo.

% \subsection{Modelo Conv3D (Modelo Convencional)}
% O Modelo Conv3D, projetado especificamente para o processamento de objetos tridimensionais, destaca-se por sua eficácia em vídeos, assemelhando-se aos modelos de Redes Neurais Recorrentes (RNN) em termos de desempenho. Contudo, sua aplicabilidade é limitada devido à exigência de um processamento computacional intensivo. A seguir, apresentamos uma visão detalhada deste modelo, incluindo a estrutura de suas camadas, as funções de ativação utilizadas e o número de neurônios em cada segmento:


% \subsection{Modelo Flatten (Modelo Eficiente)}
% O Modelo Flatten representa um marco revolucionário na área de processamento de vídeo. Este modelo sequencial, notável por sua eficiência e rapidez, se destaca pela maneira inovadora com que coleta e processa dados. A arquitetura de suas camadas em rede neural é projetada para maximizar a eficiência, permitindo aos usuários reconfigurar e retreinar a rede em questão de segundos para incorporar novos movimentos. Esta capacidade de adaptação rápida é um grande avanço, tornando o Modelo Flatten uma ferramenta poderosa para aplicações dinâmicas e em tempo real.

\section{Neural Network Models}
For the development of this project, two advanced neural network models were elaborated. The first model employs three-dimensional convolution layers (Conv3D), which are exceptionally suitable for processing video data, owing to their ability to capture both spatial and temporal characteristics. The second model, of a sequential nature, is focused on optimizing performance. This model incorporates the scientific core of the project, demonstrating an innovative and efficient approach in the processing and analysis of video data.

\subsection{Conv3D Model (Conventional Model)}
The Conv3D Model, specifically designed for the processing of three-dimensional objects, stands out for its effectiveness in videos, resembling Recurrent Neural Network (RNN) models in terms of performance. However, its applicability is limited due to the requirement of intensive computational processing. Next, we present a detailed overview of this model, including the structure of its layers, the activation functions used, and the number of neurons in each segment:

\subsection{Flatten Model (Efficient Model)}
The Flatten Model represents a revolutionary milestone in the field of video processing. This sequential model, notable for its efficiency and speed, stands out for the innovative way it collects and processes data. The architecture of its neural network layers is designed to maximize efficiency, allowing users to reconfigure and retrain the network in a matter of seconds to incorporate new movements. This ability for rapid adaptation is a significant advancement, making the Flatten Model a powerful tool for dynamic and real-time applications.

% Fim da seção dos modelos de rede neural

\section{Conclusion}
"Barium" stands as a significant milestone in human-computer interaction. The successful integration of neural networks and hand-tracking technologies not only demonstrates the technical feasibility of such innovations but also opens new avenues for intuitive and accessible interfaces. The project extends its impact beyond technology, influencing fields like education, healthcare, and entertainment.

\bibliographystyle{unsrt}
\bibliography{references}

\end{document}
